\documentclass[../notes.tex]{subfiles}

\begin{document}
	\ifSubfilesClassLoaded{\setcounter{chapter}{4}}{}
	\chapter{Computer Organization}
		The parts that make up a computer can be divided into three broad categories: the \concept{entral processing unit (CPU)}, the \concept{main memory}, and the \concept{input/output subsystem}.
		\section{Central Processing Unit}
			\begin{definition}{Central Processing Unit (CPU)}
				Performs operations on data. In most architectures, it has three parts: an \concept{Arithmetic Logic Unit (ALU)}, a \concept{control unit}, and a set of \concept{registers}.
			\end{definition}
			\subsection{Arithmetic Logic Unit}
				\begin{definition}{Arithmetic Logic Unit}
					Performs logic, shift, and arithmetic operations on data.
				\end{definition}
			\subsection{Registers}
				\begin{definition}{Register}
					Fast stand-alone storage locations that hold data temporarily.
				\end{definition}
				Different types of registers include:
				\begin{indentparagraph}
					\begin{description}
						\item[Data registers] Holds the input data and result of the operations.
						\item[Instruction registers] Holds the instructions for programs being run.
						\item[Program counter] Holds the location of the instruction currently being executed.
					\end{description}
				\end{indentparagraph}
			\subsection{Control Unit}
				\begin{definition}{Control Unit}
					Controls the operation of each subsystem using signals sent from the control unit to other subsystems.
				\end{definition}
		\section{Main Memory}
			\begin{definition}{Main Memory}
				A collection of storage locations, wach with a unique identifier called an \concept{address}. Data is transferred to and from different addresses in groups of bits called \concept{words}. A word that is $8$ bits is referred to as a \concept{byte}.
			\end{definition}
			\subsection{Address Space}
				\begin{definition}{Address Space}
					The total number of uniquely identifiable locations in memory.
				\end{definition}
				\begin{sidenote}{Memory Units}
					\begin{center}
						\begin{tabular}{llc}
							Unit & Exact Number of Bytes & Approximation\\
							\midrule
							kilobyte & $2^{10}$ $(1024)$ & $10^{3}$\\
							megabyte & $2^{20}$ $(1\,048\,576)$ & $10^{6}$\\
							gigabyte & $2^{30}$ $(1\,073\,741\,824)$ & $10^{9}$\\
							terabyte & $2^{40}$ & $10^{12}$
						\end{tabular}
					\end{center}
				\end{sidenote}
				Memory addresses are defined using unsigned binary integers.
				\begin{example}
					A computer has $32$ MB of memory. How many bits are needed to address any single byte in memory?
					\begin{enumerate}[nosep]
						\item Determine the memory space: $32$ MB is $2^{5} \times$ MB, which is $2^{5} \times 2^{20} = 2^{25}$ bytes.
						\item Determine the number of bits needed. $\log_{2}2^{25}$, which is $25$ bits.
					\end{enumerate}
				\end{example}
				\begin{example}
					A computer has $128$ MB of memory. Each word in this computer is $8$ bytes. How many bits are needed to address any single word in memory?
					\begin{enumerate}[nosep]
						\item Determine the memory space: $128$ MB is $2^{7} \times$ MB, which is $2^{27}$ bytes.
						\item Determine how many bytes each word needs. Each word is $8 (2^{3})$ bytes.
						\item Subtract the total of how many bytes each word needs from the memory space. $2^{27 - 3} = 2^{24}$
						\item Determine the number of bits needed. $\log_{2}2^{254}$, which is $24$ bits.
					\end{enumerate}
				\end{example}
			\subsection{Memory Types}
				Two main types of memory exist: \concept{RAM} and \concept{ROM}.
				\subsubsection{RAM}
					\begin{definition}{Random Access Memory (RAM)}
						Data and can be accessed randomly, without needing to access the data before it. RAM can be both read from, and written to. RAM is \concept{volatile}: the information stored inside it is lost if the computer is powered down.
					\end{definition}
					There are two broad categories of RAM: \concept{SRAM} and \concept{DRAM}.
					\begin{indentparagraph}
						\begin{description}
							\item[SRAM (Static RAM)] uses traditional \concept{flip-flop gates} to hold data. The gates hold their state, which means the data is stored as long as the power is on --- there is no need to refresh memory locations. Fast, but expensive.
							\item[DRAM (Dynamic RAM)] uses \concept{capacitors} (electrical devices that can store energy) for data storage. If the capacitor is charged, the state is $1$. If it is discharged, the state is $0$. As capacitors lose charge with time, memory locations need to be refreshed. Slow, but inexpensive.
						\end{description}
					\end{indentparagraph}
				\subsubsection{ROM}
					\begin{definition}{Read-Only Memory (ROM)}
						Like RAM, data can be accessed randomly. Unlike RAM, ROM can only be read from, it cannot be written to. ROM is \concept{nonvolatile}: the information is \emph{not} lost if the computer is powered down.
					\end{definition}
					Variations of ROM include \concept{PROM}, \concept{EPROM}, and \concept{EEPROM}.
					\begin{indentparagraph}
						\begin{description}
							\item[PROM (Programmable Read-Only Memory)] Memory is blank when the computer is shipped. The user uses special equipment to store programs on it. Once a program is stored, it cannot be overwritten.
							\item[EPROM (Erasable Programmble Read-Only Memory)] A variation of \concept{PROM}, where the user can store programs in the same way as PROM. Programs can be erased using a special device that applies \concept{ultraviolet light}. However, the EPROM needs to be physically removed.
							\item[EEPROM (Electrically Erasable Programmable Read-Only Memory)] A variation of \concept{EPROM} that uses electronic impulses to program and erase data. Does \emph{not} need to be physically removed. 
						\end{description}
					\end{indentparagraph}
			\pagebreak
			\subsection{Memory Heirarchy}
				\begin{center}
					\begin{tikzpicture}[scale=0.8]
						\coordinate (A) at (-5,0) {};
						\path (A) +(-0.5cm,0) coordinate (A1);
						\coordinate (B) at (5,0) {};
						\path (B) +(0.5cm,0) coordinate (B1);
						\coordinate (C) at (0,7) {};
						\path (C) +(-0.5cm,0) coordinate (C1);
						\path (C) +(0.5cm,0) coordinate (C2);
						\draw[name path=AC] (A) -- (C);
						\draw[name path=BC] (B) -- (C);
						\draw[<->] (A1) -- (C1) node[midway,above, sloped] {cost};
						\draw[<->] (B1) -- (C2) node[midway,above, sloped] {speed};
						\foreach \y/\A in {0/Main memory,2/Cache memory,4/Registers} {
							\path[name path=horiz] (A|-0,\y) -- (B|-0,\y);
							\draw[name intersections={of=AC and horiz,by=P},
										name intersections={of=BC and horiz,by=Q}] (P) -- (Q)
									node[midway,above] {\A};
					}
					\end{tikzpicture}
				\end{center}
				\begin{itemize}
					\item Use a small amount of costly high-speed memory where speed is crucial.
					\item Use a moderate amount of medium-speed memory to store data that is accessed often.
					\item Use a large amount of low-speed memory for data that is accessed less often.
				\end{itemize}
			\subsection{Cache Memory}
				\begin{definition}{Cache Memory}
					Memory that is faster than main memory, but slower than the CPU and its registers. Placed between the CPU and main memory. Contains a copy of a portion of main memory.
				\end{definition}
				\begin{enumerate}
					\item CPU checks the cache.
					\item If the word is there, it copies the word. Otherwise, the CPU accesses main memory, and copies a block of memory starting with the desired word. This block replaces the previous contents of cache memory.
					\item The CPU accesses the cache and copies the word.
				\end{enumerate}
		\pagebreak
		\section{Input/Output Subsystem}
			\begin{definition}{Input/Output (IO) Subsystem}
				Allows a computer to communicate with external information, and store programs and data even when the power is off. There are two types: \concept{nonstorage} and \concept{storage} devices.
			\end{definition}
			\subsection{Nonstorage Devices}
				\begin{definition}{Nonstorage device}
					A device that allows the CPU/memory to communicate with the outside world, but cannot store information.
				\end{definition}
				Examples of nonstorage devices:
				\begin{itemize}
					\item Keyboard (and other input devices)
					\item Monitor
					\item Printer
				\end{itemize}
			\subsection{Storage Devices}
				\begin{definition}{Storage device}
					A device that can store large amounts of information to be retrieved at a later stage. Cheaper than main memory. Their contents are \concept{nonvolatile}. They can also be referred to as \concept{auxilary storage devices}.
				\end{definition}
				These devices can be categorised as either \concept{magnetic}, or \concept{optical}.
				\subsubsection{Magnetic Storage Devices}
					\begin{definition}{Magnetic Storage Device}
						A device that uses magnetization to store bits of data. If a location is magnetized, it represents $1$, if it is not magentized, it represents $0$.
					\end{definition}
					Examples of magnetic storage devices:
					\begin{indentparagraph}
						\begin{description}
							\item[Magnetic Disk] One or more disks stacked on top of each other. Each disk is coated with a thin magnetic film. Information is stored and retrieved from the surface of the disk using a read/write head for each magentized surface of the disk.
								\begin{description}
									\item[Surface organization] Each surface on the disk is divided into \concept{tracks}, and each track is divided into \concept{sectors}. The tracks are separated by an \concept{intertrack gap}, and the sectors are divided by an \concept{intersector gap}.
									\item[Data access] This is a \concept{random access device} --- data can be accessed randomly without needing to access all other data located before it. The smallest storage area that can be accessed is a sector.
									\item[Performance] This depends on several factors, including the \concept{rotational speed} (how fast the disk is spinning), the \concept{seek time} (the time to move the read/write head to the desired track), and the \concept{transfer time} (the time to move data from the disk to the CPU/memory).
								\end{description}
							\item[Magnetic Tape] Tapes are mounted on two reels, and use a read/write head that reads or writes information when the tape is passed through it.
								\begin{description}
									\item[Surface organization] The width of the tape is divided into nine \concept{tracks}, each location on a track storing $1$ bit of information. Nine vertical locations can store $8$ bits of information related to a byte, plus a bit for error detection.
									\item[Data access] This is a \concept{sequential access device} --- data can only be accessed by passing through all the data located before it.
									\item[Performance] Slower than a magnetic disk, but cheaper. Often used to back up large amounts of data.  
								\end{description}
						\end{description}
					\end{indentparagraph}
				\subsubsection{Optical Storage Devices}
					\begin{definition}{Optical Storage Device}
						A device that uses laser light to store and retrieve data. Relatively recent. Followed the invention of the \concept{compact disk (CD)} to store audio information. 
					\end{definition}
					Examples of optical storage devices:
					\begin{indentparagraph}
						\begin{description}
							\item[CD-ROM (Compact disk read-only memory)] Same technology as the audio CD, developed by Phillips and Sony for recording music. Only difference is: a CD-ROM is more robust and checks for errors.
								\begin{description}
									\item[Creation] Three steps are used to create a large number of disks.
										\begin{enumerate}[label=\alph*, nosep]
											\item A \concept{master disk} is created using a high-power infrared laser that creates bit patterns on coated plastic. The laser translates the bit patterns into \concept{pits} (holes, normally represent $0$), and \concept{lands} (no holes, normally represent $1$).
											\item From the master disk, a mold is made. In the mold, the \concept{pits} are replaced by bumps.
											\item Molten \concept{polycarbonate resin} is injected into the mold to produce the same pits as the master disk. A thin layer of aliminium is added to the polycarbonate to produce a reflective surface. A protective layer of lacquer is applied, and a label is added. 
										\end{enumerate}
									\item[Reading] The CD-ROM is read using a low-power laser beam. The beam is reflected by the aliminium surface when passing through a \concept{land}, and reflected twice when encountering a \concept{pit}. The two reflections have a destructive effect, so less light is reflected.
									\item[Format] Different from the magnetic disk.
										\begin{itemize}[nosep]
											\item A block of $8$-bit data transformed into a $14$-bit \concept{symbol} using an error-correction method called \concept{Hamming code}.
											\item A \concept{frame} made up from $42$ symbols.
											\item A \concept{sector} made up from $98$ frames ($2352$ bytes).
										\end{itemize}
									\pagebreak
									\item[Speed] These drives come in different speeds, which are referred to as $1$x, $2$x, and so on.
										\begin{center}
											\begin{tabular}{lrl}
												Speed & Data rate & Approximation\\
												\midrule
												$1$x  & $153\,600$ bytes per second    & $150$ kB/s\\
												$2$x  & $307\,200$ bytes per second    & $300$ kB/s\\
												$4$x  & $614\,400$ bytes per second    & $600$ kB/s\\
												$6$x  & $921\,600$ bytes per second    & $900$ kB/s\\
												$8$x  & $1\,228\,800$ bytes per second & $1.2$ MB/s\\
												$12$x & $1\,843\,200$ bytes per second & $1.8$ MB/s\\
												$16$x & $2\,457\,600$ bytes per second & $2.4$ MB/s\\
												$24$x & $3\,688\,400$ bytes per second & $3.6$ MB/s\\
												$32$x & $4\,915\,200$ bytes per second & $4.8$ MB/s\\
												$40$x & $6\,144\,000$ bytes per second & $6$ MB/s
											\end{tabular}
										\end{center}
									\item[Application] Economical if the disks are mass produced. 
								\end{description}
							\item[CD-R (Compact Disk Recordable)] Users can create one or more disks without needing to go through the expense involved in making CD-ROMs. Can be written to once, but can be read many times. This is called \concept{write once, read many (WORM)}.
								\begin{description}
									\item[Creation] Uses the same principles as CD-ROM, but
										\begin{enumerate}[label=\alph*, nosep]
											\item There is no master disk or mold
											\item The reflective layer is made of gold instead of aliminium
											\item There are no physical pits in the polycarbonate: the pits and lands are simulated. This is done by adding an extra layer of dye between the reflective material and the polycarbonate.
											\item A high-power laser beam makes a dark spot in the dye, changing the chemical composition, which simulates a pit. The areas not struck simulate lands.
										\end{enumerate} 
									\item[Reading] Can be read by a CD-ROM or CD drive.
									\item[Format] Format, capacity and speed are the same as CD-ROMs.
									\item[Application] Useful for creation and distribution of a small number of disks, as well as backups.
								\end{description}
							\item[CD-RW (Compact Disk Rewritable)] As CD-Rs can only be written to once, this technology was developed. Also called \concept{erasable optical disk}. It allows contents to be written and read from the disk multiple times.
								\begin{description}
									\item[Creation] Uses the same principles as CD-R, but
										\begin{enumerate}[label=\alph*, nosep]
											\item Instead of dye, an alloy of silver, indium, antimony and telurium is used. This alloy has two stable states: \concept{crystalline} (transparent) and \concept{amorphous} (nontransparent).
											\item High-power lasers are uses to create simulated pits in the alloy. A simulated pit is amorphous, a land is crystalline.
										\end{enumerate}
									\item[Reading] Uses the same low-power laser beam that CD-ROM and CD-R use.
									\item[Erasing] A medium-power laser beam changes pits to lands, from amorphous to crystalline.
									\item[Format] Format, capacity and speed are the same as CD-ROMs and CD-Rs.
									\item[Application] More expensive than CD-Rs. It can be preferable to use a CD-R instead of a CD-RW if the data on the disk must not be changed.
								\end{description}
							\pagebreak
							\item[DVD (Digital Versatile Disk)] Improved storage quality over CDs. Uses similat technology to CD-ROM, except
								\begin{enumerate}[label=\alph*, nosep]
									\item The pits are smaller: $0.4$ microns instead of $0.8$ used in CDs.
									\item The tracks are closer to each other.
									\item The beam is a red laser instead of infrared.
									\item DVDs use one to two recording layers, and can be single-sided or double-sided.
								\end{enumerate}
								\begin{description}
									\item[Capacity] These changes result in higher capacities:
										\begin{center}
											\begin{tabular}{ll}
												Feature & Capacity\\
												\midrule
												Single-sided, single-layer & $4.7$ GB\\
												Single-sided, dual-layer & $8.5$ GB\\
												Double-sided, single-later & $9.4$ GB\\
												Double-sided, dual-layer & $17$ GB
											\end{tabular}
										\end{center}
									\item[Compression] Uses MPEG for compression.
									\item[Application] Many applications that need to store a high volume of data.
								\end{description}
						\end{description}
					\end{indentparagraph}
		\section{Subsystem Interconnection}
			\subsection{Connecting CPU and Memory}
				The CPU and memory are connected by three groups of connections, each called a \concept{bus}: data bus, address bus, and control bus.
					\begin{definition}{Data Bus}
						Made of several connections, each carrying $1$ bit at a time. The number of connections is dependent on the size of the word used by the computer.
					\end{definition}
					\begin{definition}{Address Bus}
						Allows access to a particular word in memory. The number of connrections depends on the address space of the memory. If the memory has $2^{n}$ words, the address bus needs to carry $n$ bits at a time, so must have $n$ connections.
					\end{definition}
					\begin{definition}{Control Bus}
						Carries communication between the CPU and memory. The number of connections is dependent on the total number of control commands a computer needs.
					\end{definition}
			\pagebreak
			\subsection{Connecting I/O Devices}
				I/O devices are attached to the buses through \concept{input/output controllers} or interfaces. There is one specific controller for each I/O device.
				\subsubsection{Controllers}
					\begin{definition}{Controller}
						Bridge the gap between the nature of the I/O device, the CPU, and memory. Can be either \concept{serial} or \concept{parallel}.
						\begin{indentparagraph}
							\begin{description}
								\item[Serial Controller] Has only one data wire
								\item[Parallel Controller] Has several data connections, so several bits can be transferred at once.
							\end{description}
						\end{indentparagraph}
					\end{definition}
					Some examples of controllers are:
					\begin{indentparagraph}
						\begin{description}
							\item[SCSI (Small Computer System Interface)] First developed for Macintosh computers in $1984$. A parallel interface with $8$, $16$, or $32$ connections. Provides a \concept{daisy-chained connection}. Both ends of the chain must be connected to a special device called a \concept{terminator}, and each device must have a unique address.
							\item[FireWire] Defined by IEEE standard $1394$. A high-speed serial interface that transfers data in packets, achieving a transfer rate of up to $50$ MB/s. Can connect up to $63$ devices in a daisy chain or a tree connection. No need for termination.
							\item[USB (Universal Serial Bus)] Competitor for FireWire. A serial controller than connects both low and high speed devices to the computer bus.
								\begin{itemize}
									\item Multiple devices can be connected to a USB controller, which ic also called a \concept{root hub}. USB-$2$ allows up to $127$ devices to be connected in a tree-like structure. The controller is the root of the tree, different \concept{hubs} work as intermediate nodes, and the devices are the end nodes.
									\item Devices can be removed or attached without powering down the computer. This is called \concept{hot-swappable}.
									\item Use a cable with $4$ wires. $2$ wires are used to provide power for low power devices. The other $2$ wires, which are twisted together to reduce noise, are used to carry data, addresses, and control signals.
									\item USB-2 provides different data transfer rates: $1.5$ Mbps (megabits per second), $12$ Mbps, and $480$ Mbps.
									\item Data is transferred over USB in \concept{packets}. Each packet contains an address part, a control part, and part of the data to be transmitted to the device.
								\end{itemize}
							\item[HDMI (High-Definition Multimedia Interface)] Digital replacement for existing analogue video standards. Used to transfer digital video or audio data from a source to a compatible display device.
						\end{description}
					\end{indentparagraph}
				\subsubsection{Addressing Input/Output Devices}
					There are two methods for handling the addressing of I/O devices: \concept{isolated I/O} and \concept{memory-mapped I/O}.
					\begin{definition}{Isolated I/O}
						Instructions used to read/write memory are totally different than the instructions used to read/write I/O devices. Each I/O device has its own address. The I/O addresses can overlap with the memory addresses without ambiguity, as the instructions are different.
						\begin{description}
							\item[Advantages] Lack of ambiguity regarding instructions.
							\item[Disadvantages] A larger number of unique instructions are needed in the CPU.
						\end{description}
					\end{definition}
					\begin{definition}{Memory-Mapped I/O}
						The CPU treats each register in the I/O controller as a word in memory. There are no separate instructions for transferring data from memory, and transferring data from I/O devices.
						\begin{description}
							\item[Advantages] A smaller number of unique instructions is needed.
							\item[Disadvantages] Part of the memory address space is allocation to registers in I/O controllers.
						\end{description}
					\end{definition}
			\section{Program Execution}
				A set of instructions called a \concept{program} is used to process data. Both the program and the data for the program are stored in memory.
				\subsection{Machine Cycle}
					\begin{definition}{Machine Cycle}
						A cycle that is repeated by the CPU, in order to execute instructions from a program.
					\end{definition}
					Contains three phases:
					\begin{enumerate}[nosep]
						\item fetch
						\item decode
						\item execute
					\end{enumerate}
					\begin{description}
						\item[Fetch] Control unit orders the system to copy the next instructions into the instruction register of the CPU. The \concept{program counter} holds the address of the instruction to be copied, and is incremented once the instruction is copied.
						\item[Decode] Translate the instruction into binary code for some operation that the computer must perform.
						\item[Execute] Control unit sends the decoded task to a component in the CPU.  
					\end{description}
				\pagebreak
				\subsection{Input/Output Operation}
					Commands are required to transfer data from I/O devices to the CPU and memory. As these devices operate at slower speeds, there needs to be a way to synchronise the operation of the CPU with the operation of the device. There are three methods:
					\begin{enumerate}[nosep]
						\item Programmed I/O
						\item Interrupt-Driven I/O
						\item Direct Memory Access
					\end{enumerate}
					\begin{definition}{Programmed I/O}
						The CPU waits for the I/O device. When the CPU encounters an I/O instruction, it does nothing else until the data transfer is complete. The CPU constantly checks the status of the IO device.
						\begin{description}
							\item[Issue:] CPU time is wasted by checking the status of the I/O for each unit of data to be transferred.
						\end{description}
						Data is tranferred to memory after the input operation, and transferred from memory before the output operation.
					\end{definition}
					\begin{definition}{Interrupt-Driven I/O}
						CPU informs the I/O device that a transfer is going to happen. CPU \emph{does not} continuously test the status of the I/O device. The I/O device informs the CPU when it is ready.
						\par Data is tranferred to memory after the input operation, and transferred from memory before the output operation.
					\end{definition}
					\begin{definition}{Direct Memory Access (DMA)}
						Transfers a large of block of data between a high-speed I/O device, and memory directly, without passing it through the CPU. Requires a \concept{DMA controller} that relieves the CPU of some of its functions. Ther controller has registers to hold blocks of data before and after memory transfer.
						\begin{enumerate}
							\item The CPU sends a message to the DMA. The message contains the type of transfer, the beginning addresses of the memory location, and the number of bytes to be transferred.
							\item The DMA controller (when ready) informs the CPU that it needs to take control of the buses. The CPU stops using the buses, and the controller uses them.
							\item After data transfer, the CPU continues normal operation.
						\end{enumerate}
						\begin{description}
							\item[Issue:] The CPU is idle for a time. This duration is short, though, as it is only during the transfer itself --- not while the data is being prepared.
						\end{description}
					\end{definition}
				\pagebreak
			\section{Different Architectures}
				\begin{definition}{Complex Instruction Set Computer (CISC)}
					Have a large set of instructions, including complex ones.

					Easier to program for than other architectures, because there is a single instruction for both simple and complex tasks. Multiple instructions don't need to be written for a complex task.

					The circuity of the CPU and control unit is very complicated. To simplify this, programming is done on two levels. The CPU only performs simple operations, called \concept{microoperations}. A complex instruction is transformed into a set of these simple operations, and then executed by the CPU. The instructions for each complex instruction are held in a special memory called \concept{micromemory}. The type of programming that uses this is called \concept{microprogramming}.

					\begin{description}
						\item[Issue] The overhead associated with microprogramming and access to micromemory.
						\item[Example] Intel Pentium series
					\end{description}
				\end{definition}
				\begin{definition}{Reduced Instruction Set Computer (RISC)}
					Have a small set of instructions that do a minimum number of simple operations.

					Complex operations are simulated using a subset of simple operations. Because of this, programming is more difficult and time-consuming.
				\end{definition}

		\rulechapterend

\end{document}