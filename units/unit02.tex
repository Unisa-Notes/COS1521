\providecommand{\main}{..}
\documentclass[\main/notes.tex]{subfiles}

\begin{document}
	\setcounter{chapter}{1}
	\chapter{Number Systems}
		A \concept{number system} defines how a number can be represented using distinct symbols.
		\section{Positional Number Systems}
			In a \concept{positional number system}, the position a symbol occupies in the number determins the value it represents.
			\begin{description}
				\item[Maximum Value] To calculate the highest possible value that a specific positional number system can have based on the number of digits $K$:
				\begin{align*}
					N_{\max} &= b^{K} - 1
				\end{align*}
			\end{description}
			\subsection{Decimal System}
				This system uses $10$ as a base. The possible symbols are $\{0, 1, 2, 3, 4, 5, 6, 7, 8, 9\}$.\\
				Therefore the number $234$ would be 
				\begin{align*}
					\bigl(234\bigr)_{10} &= \left(2 \times 10^{2}\right) + \left(3 \times 10^{1}\right) + \left(4\times 10^{0}\right)
				\end{align*}
				\subsubsection{Maximum Value}
					Given the number of digits $K$
					\begin{align*}
						N_{\max} &= b^{K} - 1\\
						&= 10^{K} - 1
					\end{align*}
					For example, suppose there are $3$ positions.
					\begin{align*}
						N_{\max} &= 10^{3} - 1\\
						&= 1000 - 1\\
						&= 999
					\end{align*}
					So the highest number that can be represented with $3$ positions is $(999)_{10}$.
			\pagebreak
			\subsection{Binary System}
				This system uses $2$ as a base. The possible symbols are $\{0, 1\}$. The number $10101$ in decimal would be
				\begin{align*}
					\bigl(10101\bigr)_{2} &= \left(1 \times 2^{4}\right)_{10} + \left(0 \times 2^{3}\right)_{10} + \left(1 \times 2^{2}\right)_{10} + \left(0 \times 2^{1}\right)_{10} + \left(1 \times 2^{0}\right)_{10}\\
					&= (16)_{10} + (0)_{10} + (4)_{10} + 0_{10} + (1)_{10}\\
					&= (21)_{10}
				\end{align*}
				\subsubsection{Maximum Value}
					Given the number of digits $K$
					\begin{align*}
						N_{\max} &= b^{K} - 1\\
						&= 2^{K} - 1
					\end{align*}
					For example, suppose there are $5$ positions.
					\begin{align*}
						N_{\max} &= 2^{5} - 1\\
						&= (32)_{10} - (1)_{10}\\
						&= (31)_{10}
					\end{align*}
					So the highest number that can be represented with $5$ positions is $(31)_{10}$, which would be $(11111)_{2}$.
				\subsubsection{Real Values}
					If there is a fractional part to the number, say $10.01$, then the fractional part uses $2^{-k}$, where $k$ is the position.\\
					For example:
					\begin{align*}
						\bigl(11.01\bigr)_{2} &= \left(1 \times 2^{1}\right)_{10} + \left(1 \times 2^{0}\right)_{10} + \left(0 \times 2^{-1}\right)_{10} + \left(1 \times 2^{-2}\right)_{10}\\
						&= (2)_{10} + (1)_{10} + (0)_{10} + (0.25)_{10}\\
						&= (3.25)_{10}
					\end{align*}
			\pagebreak
			\subsection{Hexadecimal System}
				This system uses $16$ as a base. The possible symbols are $\{0, 1, 2, 3, 4, 5, 6, 7, 8, 9, \mathrm{A}, \mathrm{B}, \mathrm{C}, \mathrm{D}, \mathrm{E}\}$.\\
				Therefore the number $\mathrm{A}25\mathrm{C}$ would be
				\begin{align*}
					\bigl(\mathrm{A}25\mathrm{C}\bigr)_{16} &= \left(10 \times 16^{3}\right)_{10} + \left(2 \times 16^{2}\right)_{10} + \left(5 \times 16^{1}\right)_{10} + \left(12 \times 16^{0}\right)_{10}\\
					&= (10 \times 4096)_{10} + (2 \times 256)_{10} + (5 \times 16)_{10} + (12 \times 1)_{10}\\
					&= (40960)_{10} + (512)_{10} + (80)_{10} + (12)_{10}\\
					&= 41564
				\end{align*}
				\subsubsection{Maximum Value}
					Given the number of digits $K$
					\begin{align*}
						N_{\max} &= b^{K} - 1\\
						&= 16^{K} - 1
					\end{align*}
					For example, suppose there are $2$ positions.
					\begin{align*}
						N_{\max} &= 16^{2} - 1\\
						&= (256)_{10} - (1)_{10}\\
						&= (255)_{10}
					\end{align*}
					So the highest number that can be represented with $2$ positions is $(255)_{10}$, which would be $(\mathrm{FF})_{16}$.
				\subsubsection{Real Values}
					If there is a fractional part to the number, say $\mathrm{A}12.3\mathrm{B}$, then the fractional part uses $2^{-k}$, where $k$ is the position.\\
					For example:
					\begin{align*}
						\bigl(\mathrm{A}12.3\mathrm{B}\bigr)_{16} &= \left(10 \times 16^{2}\right)_{10} + \left(1 \times 16^{1}\right)_{10} + \left(2 \times 16^{0}\right)_{10} + \left(3 \times 16^{-1}\right)_{10} + \left(11 \times 16^{-2}\right)_{10}\\
						&= (10 \times 256)_{10} + (1 \times 16)_{10} + (2 \times 1)_{10} + \left(3 \times \frac{1}{16}\right)_{10} + \left(11 \times \frac{1}{256}\right)_{10}\\
						&= (2560)_{10} + (16)_{10} + (2)_{10} + \left(\frac{3}{16}\right)_{10} + \left(\frac{11}{256}\right)_{10}\\
						&= (2578)_{10} + \left(\frac{59}{256}\right)_{10}\\
						& \approx (2578.230)_{10}
					\end{align*}
			\pagebreak
			\subsection{Octal System}
				This system uses $8$ as a base. The possible symbols are $\{0, 1, 2, 3, 4, 5, 6, 7\}$.\\
				Therefore the number $567$ would be
				\begin{align*}
					\bigl(567\bigr)_{8} &= \left(5 \times 8^{2}\right)_{10} + \left(6 \times 8^{1}\right)_{10} + \left(7 \times 8^{0}\right)_{10}\\
					&= (5 \times 64)_{10} + (6 \times 8)_{10} + (7 \times 1)_{10}\\
					&= (320)_{10} + (48)_{10} + (7)_{10}\\
					&= (375)_{10}
				\end{align*}
				\subsubsection{Maximum Value}
					Given the number of digits $K$
					\begin{align*}
						N_{\max} &= b^{K} - 1\\
						&= 8^{K} - 1
					\end{align*}
					For example, suppose there are $4$ positions.
					\begin{align*}
						N_{\max} &= 8^{4} - 1\\
						&= (4096)_{10} - (1)_{10}\\
						&= (4095)_{10}
					\end{align*}
					So the highest number that can be represented with $4$ positions is $(4095)_{10}$, which would be $(7777)_{8}$.
			\subsection{First 15 Numbers}
			\begin{table}[h]
				\centering
				\begin{tabular}{cccc}
					\toprule
					Decimal & Binary & Octal & Hexadecimal\\
					\midrule
					$01$ & $0001$ & $01$ & $1$\\
					$02$ & $0010$ & $02$ & $2$\\
					$03$ & $0011$ & $03$ & $3$\\
					$04$ & $0100$ & $04$ & $4$\\
					$05$ & $0101$ & $05$ & $5$\\
					$06$ & $0110$ & $06$ & $6$\\
					$07$ & $0111$ & $07$ & $7$\\
					$08$ & $1000$ & $10$ & $8$\\
					$09$ & $1001$ & $11$ & $9$\\
					$10$ & $1010$ & $12$ & $\mathrm{A}$\\
					$11$ & $1011$ & $13$ & $\mathrm{B}$\\
					$12$ & $1100$ & $14$ & $\mathrm{C}$\\
					$13$ & $1101$ & $15$ & $\mathrm{D}$\\
					$14$ & $1110$ & $16$ & $\mathrm{E}$\\
					$15$ & $1111$ & $17$ & $\mathrm{F}$\\
					\bottomrule
				\end{tabular}
			\end{table}
		\pagebreak
		\section{Conversion}
			\subsection{Any base to decimal}
				Multiply each digit with its place value in the source system.\\
				All the examples above are of this type.
			\subsection{Decimal to any base}
				Use repetitive division of the base, and the remainder. Fill in the numbers from right to left.
				\begin{example}
					Given the decimal number $25$, convert to binary.\\
					Divide the top number by $2$ each time, and store the remainder below.
					\begin{center}
						\begin{tikzpicture}[node distance=1.5cm]
							\node (start) [numstart] {$25$};
							\node (num1) [numbox, left of=start] {$12$};
							\node (rem1) [numbox, below of=start] {$1$};
							\node (num2) [numbox, left of=num1] {$6$};
							\node (rem2) [numbox, below of=num1] {$0$};
							\node (num3) [numbox, left of=num2] {$3$};
							\node (rem3) [numbox, below of=num2] {$0$};
							\node (num4) [numbox, left of=num3] {$1$};
							\node (rem4) [numbox, below of=num3] {$1$};
							\node (num5) [numbox, left of=num4] {$0$};
							\node (rem5) [numbox, below of=num4] {$1$};
							\node (toplabel) [label, right of=start] {\textbf{Decimal}};
							\node (bottomlabel) [label, right of=rem1] {\textbf{Binary}};
							\draw [arrow] (start) -- (rem1);
							\draw [arrow] (start) -- (num1);
							\draw [arrow] (num1) -- (rem2);
							\draw [arrow] (num2) -- (rem3);
							\draw [arrow] (num3) -- (rem4);
							\draw [arrow] (num4) -- (rem5);
							\draw [arrow] (num1) -- (num2);
							\draw [arrow] (num2) -- (num3);
							\draw [arrow] (num3) -- (num4);
							\draw [arrow] (num4) -- (num5);
						\end{tikzpicture}
					\end{center}
					Therefore $(25)_{10}$ is $(11001)_{2}$.
				\end{example}
				\begin{example}
					Given the decimal number $76$, convert to octal.\\
					Divide the top number by $8$ each time, and store the remainder below.
					\begin{center}
						\begin{tikzpicture}[node distance=1.5cm]
							\node (start) [numstart] {$76$};
							\node (num1) [numbox, left of=start] {$9$};
							\node (num2) [numbox, left of=num1] {$1$};
							\node (num3) [numbox, left of=num2] {$0$};
							\node (toplabel) [label, right of=start] {\textbf{Decimal}};
							\node (rem1) [numbox, below of=start] {$4$};
							\node (rem2) [numbox, below of=num1] {$1$};
							\node (rem3) [numbox, below of=num2] {$1$};
							\node (bottomlabel) [label, right of=rem1] {\textbf{Octal}};
							\draw [arrow] (start) -- (num1);
							\draw [arrow] (num1) -- (num2);
							\draw [arrow] (num2) -- (num3);
							\draw [arrow] (start) -- (rem1);
							\draw [arrow] (num1) -- (rem2);
							\draw [arrow] (num2) -- (rem3);
						\end{tikzpicture}
					\end{center}
					Therefore $(76)_{10}$ is $(114)_{8}$.
				\end{example}
				\begin{example}
					Given the decimal number $216$, convert to hexadecimal.\\
					Divide the top number by $16$ each time, and store the remainder below.
					\begin{center}
						\begin{tikzpicture}[node distance=1.5cm]
							\node (start) [numstart] {$216$};
							\node (num1) [numbox, left of=start] {$13$};
							\node (num2) [numbox, left of=num1] {$0$};
							\node (toplabel) [label, right of=start, xshift=1cm] {\textbf{Decimal}};
							\node (rem1) [numbox, below of=start] {$8$};
							\node (rem2) [numbox, below of=num1] {$\mathrm{D}$};
							\node (bottomlabel) [label, right of=rem1, xshift=1cm] {\textbf{Hexadecimal}};
							\draw [arrow] (start) -- (num1);
							\draw [arrow] (num1) -- (num2);
							\draw [arrow] (start) -- (rem1);
							\draw [arrow] (num1) -- (rem2);
						\end{tikzpicture}
					\end{center}
					Therefore $(216)_{10}$ is $(\mathrm{D}8)_{16}$
				\end{example}
				\pagebreak
				\subsubsection{Fractional Part}
					Use repetitive multiplication of the base, and the remainder. Fill in the numbers from left to right.
					\begin{example}
						Given the decimal number $0.625$, convert to binary.\\
						Multiply the number by $2$ each time, and store the integer part below.
						\begin{center}
							\begin{tikzpicture}[node distance=1.5cm]
								\node (start) [numstart] {$0.625$};
								\node (num1) [numbox, right of=start] {$0.25$};
								\node (num2) [numbox, right of=num1] {$0.5$};
								\node (num3) [numbox, right of=num2] {$0.0$};
								\node (toplabel) [label, left of=start, xshift=-1cm] {\textbf{Decimal}};
								\node (rem1) [numbox, below of=start] {$1$};
								\node (rem2) [numbox, below of=num1] {$0$};
								\node (rem3) [numbox, below of=num2] {$1$};
								\node (bottomlabel) [label, left of=rem1, xshift=-1cm] {\textbf{Binary}};
								\node (point) [label, left of=rem1, xshift=0.5cm] {$\bullet$};
								\draw [arrow] (start) -- (num1);
								\draw [arrow] (num1) -- (num2);
								\draw [arrow] (num2) -- (num3);
								\draw [arrow] (start) -- (rem1);
								\draw [arrow] (num1) -- (rem2);
								\draw [arrow] (num2) -- (rem3);
							\end{tikzpicture}
						\end{center}
						Therefore $\bigl(0.625\bigr)_{10}$ is $\bigl(0.101\bigr)_{2}$
					\end{example}
					\begin{example}
						Given the decimal number $0.634$, convert to octal, using a maximum of $4$ digits.\\
						Multiply the number by $8$ each time, and store the integer part below.
						\begin{center}
							\begin{tikzpicture}[node distance=1.5cm]
								\node (start) [numstart] {$0.634$};
								\node (num1) [numbox, right of=start] {$0.072$};
								\node (num2) [numbox, right of=num1] {$0.576$};
								\node (num3) [numbox, right of=num2] {$0.608$};
								\node (num4) [numbox, right of=num3] {$0.864$};
								\node (toplabel) [label, left of=start, xshift=-1cm] {\textbf{Decimal}};
								\node (rem1) [numbox, below of=start] {$5$};
								\node (rem2) [numbox, below of=num1] {$0$};
								\node (rem3) [numbox, below of=num2] {$4$};
								\node (rem4) [numbox, below of=num3] {$4$};
								\node (bottomlabel) [label, left of=rem1, xshift=-1cm] {\textbf{Octal}};
								\node (point) [label, left of=rem1, xshift=0.5cm] {$\bullet$};
								\draw [arrow] (start) -- (num1);
								\draw [arrow] (num1) -- (num2);
								\draw [arrow] (num2) -- (num3);
								\draw [arrow] (num3) -- (num4);
								\draw [arrow] (start) -- (rem1);
								\draw [arrow] (num1) -- (rem2);
								\draw [arrow] (num2) -- (rem3);
								\draw [arrow] (num3) -- (rem4);
							\end{tikzpicture}
						\end{center}
						Therefore $\bigl(0.634\bigr)_{10}$ is approximately $\bigl(0.5044\bigr)_{8}$
					\end{example}
			\rulechapterend
\end{document}
